\documentclass[11pt, a4paper]{article}
\usepackage{graphicx}
\usepackage{amsmath}
\usepackage{listings}


\title{\textbf{Assignment-4}} % Title

\author{\textbf{\textit{M.Sai kumar}}} % Author name

\date{18-02-2022} % Date for the report

\begin{document}

\maketitle % Insert the title, author and date
\section{Plotting given functions.}
Given functions are plotted by taking 500 points in the interval $[-2\pi,4\pi)$. Function $e^x$ is plotted using semilogy plot and $cos(cos(x))$ is plotted.
By seeing the plots we can say that $e^x$ is not periodic but $cos(cos(x))$ is periodic. But both functions can be generated by fourier series(not exactly but approximatly)
Below are the plots of given functions.$e^x$ is plotted using semilogy.
The matplotlib.pyplot.semilogy() function in pyplot module of matplotlib library is used to make a plot with log scaling on the y axis.
\begin{figure}[!tbh]
  \centering
  \includegraphics[scale=0.5]{Figure1.png} 
  \caption{semilogy of $e^x$ vs x} 
  \label{fig:fig1}
\end{figure} 
\begin{figure}[!tbh]
  \centering
  \includegraphics[scale=0.5]{Figure2.png}  
  \caption{plot of $\cos(\cos(x))$ vs x} 
  \label{fig:fig2}
\end{figure} 
\pagebreak
\newpage
\section{Plotting Coefficients of fourier series.}
Given functions are changed into their respective fourier series. First 51 coefficients are calculated using inbuilt integrate function quad.
\begin{verbatim}	
    quad(u,0,2*pi,args=(k))
\end{verbatim}
The quad function returns the two values, in which the first number is the value of integral and the second value is the estimate of the absolute error in the value of integral.
The problem is that quad only knows to call our function as f(x).
If we want quad to call our function with other parameters, then we need to tell quad about them.
Use the args argument of the quad function.
We set args to be a tuple of all of our extra function arguments besides x.
\newline
a) bn coeficients of $\cos(\cos(x))$ will be zero because it is an even function. But it is nearly zero because of limitation in calculations.
\newline
b)The first function is exponentially increasing so it's fourier series will contain higher frequencies. Whereas in second case higher frequiencies doesn't contribute much.
\newline
Below plots are coefficients of fourier series of 2 functions.
\begin{figure}[!tbh]
  \centering
  \includegraphics[scale=0.5]{Figure3.png}
  \caption{(semilogy)magnitude of the coefficients vs n} 
  \label{fig:fig3}
\end{figure} 
\begin{figure}[!tbh]
  \centering
  \includegraphics[scale=0.5]{Figure4.png}  
  \caption{(loglog)magnitude of the coefficients vs n} 
  \label{fig:fig4}
\end{figure}
\pagebreak
\newpage
\begin{figure}[!tbh]
    \centering
    \includegraphics[scale=0.5]{Figure5.png}
    \caption{(semilogy)magnitude of the coefficients vs n} 
    \label{fig:fig5}
  \end{figure} 
  \begin{figure}[!tbh]
    \centering
    \includegraphics[scale=0.5]{Figure6.png}  
    \caption{(loglog)magnitude of the coefficients vs n} 
    \label{fig:fig6}
  \end{figure}
\pagebreak
\newpage
\section{Plotting cofficients of least square approach and fourier series approach.}
We need to define a vector x going from 0 to $2\pi$ in 400 steps. Then for each x we need to find fourier series.
It can be done by building a matrix of 400,51 size and solving the matrix with the function values.It can be done using $lstsq$.
\newline
Matrix can be constructed as given below.
\begin{verbatim}	
    x=linspace(0,2*pi,401)
    x=x[:-1] 
    b=f(x) 
    A=zeros((400,51)) 
    A[:,0]=1 
    for k in range(1,26):
    A[:,2*k-1]=cos(k*x)
    A[:,2*k]=sin(k*x)
    c1=lstsq(A,b)[0]
\end{verbatim}
  \begin{figure}[!tbh]
    \centering
    \includegraphics[scale=0.5]{Figure7.png}  
    \caption{(semilogy)magnitude of the coefficients vs n} 
    \label{fig:fig7}
  \end{figure}
  \begin{figure}[!tbh]
    \centering
    \includegraphics[scale=0.5]{Figure8.png}  
    \caption{(loglog)magnitude of the coefficients vs n} 
    \label{fig:fig8}
  \end{figure}
  \begin{figure}[!tbh]
    \centering
    \includegraphics[scale=0.5]{Figure9.png}  
    \caption{(semilogy)magnitude of the coefficients vs n} 
    \label{fig:fig9}
  \end{figure}
  \begin{figure}[!tbh]
    \centering
    \includegraphics[scale=0.5]{Figure10.png}  
    \caption{(loglog)magnitude of the coefficients vs n} 
    \label{fig:fig10}
  \end{figure}
\pagebreak
\newpage
\section{Plotting deviation}
After getting coefficients through least squares approach and fourier series approximation we can find difference between those coefficients.
Cofficients through both process are not equal.They should be equal but due to best fit,other approximations and limitations they deviate.
  \begin{figure}[!tbh]
    \centering
    \includegraphics[scale=0.5]{Figure11.png}  
    \caption{Deviation(absolute difference between the two sets of coefficients)} 
    \label{fig:fig11}
  \end{figure}
  \begin{figure}[!tbh]
    \centering
    \includegraphics[scale=0.5]{Figure12.png}  
    \caption{Deviation(absolute difference between the two sets of coefficients)} 
    \label{fig:fig12}
  \end{figure}
\pagebreak
\newpage
\section{Plotting Estimated and true functions.}
We can see that $e^x$ shows more deviation but $cos(cos(x))$ shows no deviation because we consider higher frequencies in case of $e^x$. In case of $cos(cos(x))$ higher frequencies doen't play much role.
  \begin{figure}[!tbh]
    \centering
    \includegraphics[scale=0.5]{Figure13.png}  
    \caption{Estimated and true function vs x} 
    \label{fig:fig13}
  \end{figure}
  \begin{figure}[!tbh]
    \centering
    \includegraphics[scale=0.5]{Figure14.png}  
    \caption{Estimated and true function vs x} 
    \label{fig:fig14}
  \end{figure}
\end{document}

